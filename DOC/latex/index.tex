P\-R\-I\-M\-A\-V\-E\-R\-A 2014.

Los científicos de un laboratorio de investigación biológica desean llevar a cabo una serie de experimentos para estudiar el ciclo de vida de una especie de organismos celulares sencillos.

Los científicos del laboratorio jugaran con organismos de células estructurados en forma de árbol. Cada célula del organismo contiene dos atributos\-: el identificador (un número natural mayor que zero) y la actividad, ya que una célula puede ser o bien activa (true) o pasiva (false).

Por cada experimento que realizen recibiran al inicio unos N organismos iniciales, con un M maximo organismos por el experimento, dónde M siempre será mas grande estrictamente que N.

Entonces a partir de estos N organismos iniciales dispondran de cinco opciones para poder trabajar con ellos\-:

Opción 1. Aplicar un estirón a un subconjunto de organismos.

Opción 2. Aplicar un recorte a un subconjunto de organismos.

Opción 3. Aplicar una ronda de reproducción en el experimento.

Opción 4. Obtener el ranking de reproducción de todos los organismos existentes.

Opción 5. Consultar el estado de un subconjunto de organismos.

Después de todo ello, el experimento finalizará cuando o bien lo finalizen manualmente, o todos los organismos hayan muerto o se haya alcanzado el máximo permitido. 